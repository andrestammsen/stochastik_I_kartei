\documentclass[a7paper,11pt,print,grid=front]{kartei}

\usepackage[T1]{fontenc}
\usepackage{ngerman}
\usepackage[latin1]{inputenc}
\usepackage{amsfonts, amsmath, amssymb}
\usepackage{bm}
\usepackage[mathscr]{euscript}
\usepackage{enumitem}
\usepackage[onehalfspacing]{setspace}
%\usepackage[singlespacing]{setspace}
%\usepackage[doublespacing]{setspace}


\begin{document}

\begin{karte}[Stochastik-I]{Eigenschaften des Ma�integrals}[2.4.4]

$ f,g \in \overline{Z\XcalA} $ und $ \XcalAmu $ Ma�raum, dg:

\begin{enumerate}[label=(\roman*)]

\item $ f = 0 \ \mae \Longrightarrow \integral{}{}{f}{\mu} = 0 $
\item $ \integral{}{}{f}{\mu} = 0 \Longleftrightarrow f = 0 \ \mae $
(falls $ f \geq 0 $)
\item $ \integral{}{}{g}{\mu} $ existiert und $ \integral{}{}{g}{\mu} = \integral{}{}{f}{\mu} \ \Longleftarrow $
\vspace{0.15 cm}
\newline
(falls $ \integral{}{}{f}{\mu} $ existiert und $ f = g \ \mae $)
\item $ f \ \mintable \ \Longrightarrow f $ ist endlich $ \mae $
\end{enumerate}

\end{karte}


\begin{karte}[Stochastik-I]{Algebraische Induktion}[2.4.15]

$ f,g \in \overline{Z\XcalA} $ und $ \XcalAmu $ Ma�raum, dg:

\begin{enumerate}[label=(\roman*)]
\item Zeige $ AUS(f) $, wobei $ f = I_A $, $ \forall A \in \calA $
\item Zeige $ AUS(f) $, wobei $ f = \alpha I_A $, $ \forall A \in \calA $
\item Zeige $ AUS(f) \ \forall f \in \EplusXcalA $
\item Zeige $ AUS(f) \ \forall f \in \ZplusXcalAinf $
\item Zeige $ AUS(f) \ \forall f \in \ZXcalAinf $
\end{enumerate}

\end{karte}


\begin{karte}[Stochastik-I]{Satz von der monotonen Konvergenz}[2.4.17]

Sei $ \XcalAmu $ Ma�raum, $ (f_n)_{n\in\R} $ eine $ \mae $ mon wachs
Folge n-neg Funktionen aus $ \ZXcalAinf  $ und $ f $ eine Funktion,
die $ \mae \ f(x) = lim_{n\to\infty} $, $ f_n (x) $ erf�llt, dg:
\cmone
$$
\integral{}{}{\lim_{n\to\infty} f_n}{\mu} = \integral{}{}{f}{\mu} = \lim_{n\to\infty} \integral{}{}{f_n}{\mu}
$$

\end{karte}


\begin{karte}[Stochastik-I]{Satz von B. Levi}[2.4.18]

Sei $ \XcalAmu $ Ma�raum, $ (f_n)_{n\in\N} $ Folge n-neg Funktionen
aus $ \ZXcalAinf $, dg:
\cmone
$$
\integral{}{}{\summ{n=1}{\infty}{} \, f_n}{\mu} = \summ{n=1}{\infty}{\integral{}{}{f_n}{\mu}}
$$

\end{karte}


\begin{karte}[Stochastik-I]{Lemma von Fatou}[2.4.19]

Sei $ \XcalAmu $ Ma�raum und $ (f_n)_{n\in\N} $ eine Folge n-neg \\
Funktionen aus $ \ZXcalAinf $, dg:
\cmone
$$
\integral{}{}{\liminf_{n\to\infty} f_n }{\mu} \ \leq \ \liminf_{n\to\infty} \integral{}{}{f_n}{\mu}
$$

\end{karte}


\begin{karte}[Stochastik-I]{Lebesgues Satz von der majorisierten Konvergenz}[2.4.20]

Sei $ \XcalAmu $ Ma�raum, $ (fn)_{n\in\N} $ Folge aus $ \ZXcalAinf $, die \\ punktweise gegen eine Funktion $ f $ konvergieren, $ g $ eine
n-neg $ \mintable $e Funktion mit der Eigenschaft, dass
$ |f_n| \leq g $, $ \mae $, dann folgt, dass $ f $ und $ f_n \ \mintable $ sind $\forall n \in \N $ und
\cmone
$$
\integral{}{}{f}{\mu} = \integral{}{}{\lim_{n\to\infty} f_n}{\mu} = \lim_{n\to\infty} \integral{}{}{f_n}{\mu}
$$

\end{karte}


\begin{karte}[Stochastik-I]{Transformationssatz f�r Ma�integrale}[2.4.21]

Sei $ T:\XcalAmu \longrightarrow \YcalB $ eine $ \calA - \calB $ mb Abbildung und $ (Y,\calB, \mu_{T}) $ das induzierte Bildma�. Ferner sei $ f \in \ZYcalB $ eine numerische Abbildung. Dann ist $ f \circ T \in \ZXcalAinf $, dg $ \forall B \in \calB $ (falls mind. ein Integral existiert):
\cmhalf
$$
\integral{T^{-1}(B)}{}{f \circ T}{\mu} = \integral{B}{}{f}{\mu_T}
$$
\end{karte}


\begin{karte}[Stochastik-I]{Standardnormalverteilung}[unter 2.4.8]
\cmone
$$
\xi \sim \calN (1,0)
$$
\vspace{0.5 cm}
$$
\integral{A}{}{\frac{1}{\sqrt{2\pi}}\exp(\frac{-x^{2}}{2})}{x}
$$

\end{karte}



\begin{karte}[Stochastik-I]{Radon-Nikodym Dichte}[2.4.22]
Sei $ \XcalAmu $ Ma�raum und $ f \in \ZplusXcalAinf $. Dann wird wie unten ein neues Ma� $ \XcalAnu $ definiert. Die Funktion $ f $ hei�t \textbf{$ \mu $-Dichte von $ \nu $} oder \textbf{Radon-Nikodym Dichte} von $ \nu $ bzgl $ \mu $:
\cmhalf
$$
\nu: \ \calA \ni A \longrightarrow \nu(A) := \integral{A}{}{f}{\mu} \equiv \integral{}{}{f \, I_A}{\mu}
$$
\newline
Schreibweise: $ d \nu = f d \mu $ (bzw. $ \nu = f \, \mu $, $ f = \frac{d\nu}{d\mu}) $

\end{karte}


\begin{karte}[Stochastik-I]{Name gesucht :)}[2.4.23]

Sei $ \XcalAmu $ Ma�raum, $ f \in \ZplusXcalAinf $, $ d\nu := f \, d\mu $, $ \xi \in \ZXcalAinf $ eine mb Funktion, dg:
\vspace{0.5 cm}
$$
\xi \in \LoneXcalAnu \Longleftrightarrow \xi \cdot f \in \LoneXcalAmu
$$
	
\end{karte}


\begin{karte}[Stochastik-I]{Satz von Radon-Nikodym}[2.4.24]

Seien $ \mu $, $ \nu $ Ma�e auf dem mb Raum $ \XcalA$, $ \mu \ \sigma $-finit, dann sind �quivalent:
\cmhalf
\begin{enumerate}[label=(\roman*)]
\item $ \nu $ besitzt eine Dichte bzgl $ \mu $
\item $ \forall A \in \calA$: $ \mu(A) = 0 \Longrightarrow \nu(A) = 0 $
\end{enumerate}

\end{karte}


\begin{karte}[Stochastik-I]{absolut-stetig \\ \& \\ singul�r}[2.4.25]

$ \XcalA $ mb Raum, Ma�e $\mu $ und $ \nu $:
\cmhalf
\begin{enumerate}[label=(\roman*)]
\item $ \nu $ besitzt eine Dichte bzgl $ \mu $
\item $ \forall A \in \calA $: $ \mu(A) = 0 \Longrightarrow \nu(A) = 0 $
\end{enumerate}

\end{karte}


\begin{karte}[Stochastik-I]{Zerlegungssatz von Lebesgue}[2.4.25]
$ \XcalA $ mb Raum, Ma�e $ \mu $ und $ \nu $:
\cmone
\begin{itemize}
\item $ \nu $ l�sst sich auf genau eine Weise als $ \nu = \nu_1 + \nu_2 $ darstellen
\item $ \nu_1 $ und $ \nu_2 $ sind Ma�e auf $ \XcalA $
\item $ \nu_1 \ll \mu $
\item $ \nu_2 \perp \mu $
\end{itemize}
	
\end{karte}


\begin{karte}[Stochastik-I]{}[2.4.23]
	
\end{karte}

\end{document}