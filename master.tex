\documentclass[a7paper,11pt,print,grid=front]{kartei}

\usepackage[T1]{fontenc}
\usepackage{ngerman}
\usepackage[latin1]{inputenc}
\usepackage{amsfonts, amsmath, amssymb}
\usepackage[mathscr]{euscript}
\usepackage{enumitem}
\usepackage[onehalfspacing]{setspace}
%\usepackage[singlespacing]{setspace}
%\usepackage[doublespacing]{setspace}


\begin{document}

\begin{karte}[Stochastik-I]{Eigenschaften des Ma�integrals}[2.4.4]

$ f,g \in \overline{Z\XcalA} $ und $ \XcalAmu $ Ma�raum, d.g.:

\begin{enumerate}[label=(\roman*)]

\item $ f = 0 \ \mae \Longrightarrow \integral{}{}{f}{\mu} = 0 $

\item $ \integral{}{}{f}{\mu} = 0 \Longleftrightarrow f = 0 \ \mae $
(falls $ f \geq 0 $)
	
\item $ \integral{}{}{g}{\mu} $ existiert und $ \integral{}{}{g}{\mu} = \integral{}{}{f}{\mu} \ \Longleftarrow $
\vspace{0.15 cm}
\newline
(falls $ \integral{}{}{f}{\mu} $ existiert und $ f = g \ \mae $)
	
\item $ f \ \mintable \ \Longrightarrow f $ ist endlich $ \mae $

\end{enumerate}

\end{karte}


\begin{karte}[Stochastik-I]{Algebraische Induktion}[2.4.15]

$ f,g \in \overline{Z\XcalA} $ und $ \XcalAmu $ Ma�raum, d.g.:

\begin{enumerate}[label=(\roman*)]

\item Zeige $ AUS(f) $, wobei $ f = I_A $, $ \forall A \in \calA $

\item Zeige $ AUS(f) $, wobei $ f = \alpha I_A $, $ \forall A \in \calA $

\item Zeige $ AUS(f) \ \forall f \in \EplusXcalA $

\item Zeige $ AUS(f) \ \forall f \in \ZplusXcalAinf $

\item Zeige $ AUS(f) \ \forall f \in \ZXcalAinf $
\end{enumerate}

\end{karte}


\begin{karte}[Stochastik-I]{Satz von der monotonen Konvergenz}[2.4.17]

Sei $ \XcalAmu $ Ma�raum, $ (f_n)_{n\in\R} $ eine $ \mae $ mon wachs \\
Folge n-neg Funktionen aus $ \ZXcalAinf  $ und $ f $ eine Funktion, \\
die $ \mae \ f(x) = lim_{n\to\infty} $, $ f_n (x) $ erf�llt, d.g.:

$$ \integral{}{}{\lim_{n\to\infty} f_n}{\mu} = \integral{}{}{f}{\mu} = \lim_{n\to\infty} \integral{}{}{f_n}{\mu} $$

\end{karte}


\begin{karte}[Stochastik-I]{Satz von B. Levi}[2.4.18]

Sei $ \XcalAmu $ Ma�raum, $ (f_n)_{n\in\N} $ Folge n-neg Funktionen \\
aus $ \ZXcalAinf $, d.g.:

$$ \integral{}{}{\summ{n=1}{\infty}{} \, f_n}{\mu} = \summ{n=1}{\infty}{\integral{}{}{f_n}{\mu}} $$

\end{karte}


\begin{karte}[Stochastik-I]{Lemma von Fatou}[2.4.19]

Sei $ \XcalAmu $ Ma�raum und $ (f_n)_{n\in\N} $ eine Folge n-neg \\
Funktionen aus $ \ZXcalAinf $, d.g.:

$$ \integral{}{}{\liminf_{n\to\infty} f_n }{\mu} \ \leq \ \liminf_{n\to\infty} \integral{}{}{f_n}{\mu} $$

\end{karte}


\begin{karte}[Stochastik-I]{Lebesgues Satz von der majorisierten Konvergenz}[2.4.20]

Sei $ \XcalAmu $ Ma�raum, $ (fn)_{n\in\N} $ Folge aus $ \ZXcalAinf $, die \\ punktweise gegen eine Funktion $ f $ konvergieren, $ g $ eine \\
n-neg $ \mintable $e Funktion mit der Eigenschaft, dass \\
$ |f_n| \leq g $, $ \mae $, dann folgt, dass $ f $ und $ f_n \ \mintable $ \\
sind $\forall n \in \N $ und \\

$$ \integral{}{}{f}{\mu} = \integral{}{}{\lim_{n\to\infty} f_n}{\mu} = \lim_{n\to\infty} \integral{}{}{f_n}{\mu} $$

\end{karte}


\begin{karte}[Stochastik-I]{Lebesgues Satz von der majorisierten Konvergenz}[2.4.20]
	
\end{karte}


\begin{karte}[Stochastik-I]{}[]
	
\end{karte}


\begin{karte}[Stochastik-I]{}[]
	

\end{karte}

\begin{karte}[Stochastik-I]{}[]

	
\end{karte}


\begin{karte}[Stochastik-I]{}[]
	
\end{karte}


\begin{karte}[Stochastik-I]{}[]
	

\end{karte}

\begin{karte}[Stochastik-I]{}[]

	
\end{karte}


\begin{karte}[Stochastik-I]{}[]
	
\end{karte}


\begin{karte}[Stochastik-I]{}[]
	

\end{karte}

\begin{karte}[Stochastik-I]{}[]

	
\end{karte}


\begin{karte}[Stochastik-I]{}[]
	
\end{karte}


\begin{karte}[Stochastik-I]{}[]
	

\end{karte}

\begin{karte}[Stochastik-I]{}[]

	
\end{karte}


\begin{karte}[Stochastik-I]{}[]
	
\end{karte}


\begin{karte}[Stochastik-I]{}[]
	

\end{karte}

\begin{karte}[Stochastik-I]{}[]
	
\end{karte}


\begin{karte}[Stochastik-I]{}[]
	
\end{karte}


\begin{karte}[Stochastik-I]{}[]
	

\end{karte}

\begin{karte}[Stochastik-I]{}[]

	
\end{karte}


\end{document}