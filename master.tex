\documentclass[a7paper,11pt,print,grid=front]{kartei}
\usepackage[T1]{fontenc}
\usepackage{ngerman}
\usepackage[latin1]{inputenc}
\usepackage{amsfonts, amsmath, amssymb, mathtools}
\usepackage{bm}
\usepackage[mathscr]{euscript}
\usepackage{enumitem}
\usepackage[onehalfspacing]{setspace}
%\usepackage[singlespacing]{setspace}
%\usepackage[doublespacing]{setspace}


\begin{document}

\begin{karte}[Stochastik-I]{Eigenschaften des Ma�integrals}[2.4.4]

$ f,g \in \overline{Z\XcalA} $ und $ \XcalAmu $ Ma�raum, dg:

\begin{enumerate}[label=(\roman*)]

\item $ f = 0 \ \mae \Longrightarrow \intt{}{}{f}{\mu} = 0 $
\item $ \intt{}{}{f}{\mu} = 0 \Longleftrightarrow f = 0 \ \mae $
(falls $ f \geq 0 $)
\item $ \intt{}{}{g}{\mu} $ existiert und $ \intt{}{}{g}{\mu} = \intt{}{}{f}{\mu} \ \Longleftarrow $
\vspace{0.15 cm}
\newline
(falls $ \intt{}{}{f}{\mu} $ existiert und $ f = g \ \mae $)
\item $ f \ \mintable \ \Longrightarrow f $ ist endlich $ \mae $
\end{enumerate}

\end{karte}


\begin{karte}[Stochastik-I]{Algebraische Induktion}[2.4.15]

$ f,g \in \overline{Z\XcalA} $ und $ \XcalAmu $ Ma�raum, dg:

\begin{enumerate}[label=(\roman*)]
\item Zeige $ AUS(f) $, wobei $ f = I_A $, $ \forall A \in \calA $
\item Zeige $ AUS(f) $, wobei $ f = \alpha I_A $, $ \forall A \in \calA $
\item Zeige $ AUS(f) \ \forall f \in \EplusXcalA $
\item Zeige $ AUS(f) \ \forall f \in \ZplusXcalAinf $
\item Zeige $ AUS(f) \ \forall f \in \ZXcalAinf $
\end{enumerate}

\end{karte}


\begin{karte}[Stochastik-I]{Satz von der monotonen Konvergenz}[2.4.17]

Sei $ \XcalAmu $ Ma�raum, $ (f_n)_{n\in\R} $ eine $ \mae $ mon wachs
Folge n-neg Funktionen aus $ \ZXcalAinf  $ und $ f $ eine Funktion,
die $ \mae \ f(x) = lim_{n\to\infty} $, $ f_n (x) $ erf�llt, dg:
\cmone
$$
\intt{}{}{\lim_{n\to\infty} f_n}{\mu} = \intt{}{}{f}{\mu} = \lim_{n\to\infty} \intt{}{}{f_n}{\mu}
$$

\end{karte}


\begin{karte}[Stochastik-I]{Satz von B. Levi}[2.4.18]

Sei $ \XcalAmu $ Ma�raum, $ (f_n)_{n\in\N} $ Folge n-neg Funktionen
aus $ \ZXcalAinf $, dg:
\cmone
$$
\intt{}{}{\summ{n=1}{\infty}{} \, f_n}{\mu} = \summ{n=1}{\infty}{\intt{}{}{f_n}{\mu}}
$$

\end{karte}


\begin{karte}[Stochastik-I]{Lemma von Fatou}[2.4.19]

Sei $ \XcalAmu $ Ma�raum und $ (f_n)_{n\in\N} $ eine Folge n-neg \\
Funktionen aus $ \ZXcalAinf $, dg:
\cmone
$$
\intt{}{}{\liminf_{n\to\infty} f_n }{\mu} \ \leq \ \liminf_{n\to\infty} \intt{}{}{f_n}{\mu}
$$

\end{karte}


\begin{karte}[Stochastik-I]{Lebesgues Satz von der majorisierten Konvergenz}[2.4.20]

Sei $ \XcalAmu $ Ma�raum, $ (fn)_{n\in\N} $ Folge aus $ \ZXcalAinf $, die \\ punktweise gegen eine Funktion $ f $ konvergieren, $ g $ eine
n-neg $ \mintable $e Funktion mit der Eigenschaft, dass
$ |f_n| \leq g $, $ \mae $, dann folgt, dass $ f $ und $ f_n \ \mintable $ sind $\forall n \in \N $ und
\cmone
$$
\intt{}{}{f}{\mu} = \intt{}{}{\lim_{n\to\infty} f_n}{\mu} = \lim_{n\to\infty} \intt{}{}{f_n}{\mu}
$$

\end{karte}


\begin{karte}[Stochastik-I]{Transformationssatz f�r Ma�integrale}[2.4.21]

Sei $ T:\XcalAmu \longrightarrow \YcalB $ eine $ \calA - \calB $ mb Abbildung und $ (Y,\calB, \mu_{T}) $ das induzierte Bildma�. Ferner sei $ f \in \ZYcalB $ eine numerische Abbildung. Dann ist $ f \circ T \in \ZXcalAinf $, dg $ \forall B \in \calB $ (falls mind. ein Integral existiert):
\cmhalf
$$
\intt{T^{-1}(B)}{}{f \circ T}{\mu} = \intt{B}{}{f}{\mu_T}
$$
\end{karte}


\begin{karte}[Stochastik-I]{Standardnormalverteilung}[unter 2.4.8]
\cmone
$$
\xi \sim \calN (1,0)
$$
\vspace{0.5 cm}
$$
\intt{A}{}{\frac{1}{\sqrt{2\pi}}\exp(\frac{-x^{2}}{2})}{x}
$$

\end{karte}



\begin{karte}[Stochastik-I]{Radon-Nikodym Dichte}[2.4.22]
Sei $ \XcalAmu $ Ma�raum und $ f \in \ZplusXcalAinf $. Dann wird wie unten ein neues Ma� $ \XcalAnu $ definiert. Die Funktion $ f $ hei�t \textbf{$ \mu $-Dichte von $ \nu $} oder \textbf{Radon-Nikodym Dichte} von $ \nu $ bzgl $ \mu $:
\cmhalf
$$
\nu: \ \calA \ni A \longrightarrow \nu(A) := \intt{A}{}{f}{\mu} \equiv \int{}{}{f \, I_A}{\mu}
$$
\newline
Schreibweise: $ d \nu = f d \mu $ (bzw. $ \nu = f \, \mu $, $ f = \frac{d\nu}{d\mu}) $

\end{karte}


\begin{karte}[Stochastik-I]{Name gesucht :)}[2.4.23]

Sei $ \XcalAmu $ Ma�raum, $ f \in \ZplusXcalAinf $, $ d\nu := f \, d\mu $, $ \xi \in \ZXcalAinf $ eine mb Funktion, dg:
\vspace{0.5 cm}
$$
\xi \in \LoneXcalAnu \Longleftrightarrow \xi \cdot f \in \LoneXcalAmu
$$
	
\end{karte}


\begin{karte}[Stochastik-I]{Satz von Radon-Nikodym}[2.4.24]

Seien $ \mu $, $ \nu $ Ma�e auf dem mb Raum $ \XcalA$, $ \mu \ \sigma $-finit, dann sind �quivalent:
\cmhalf
\begin{enumerate}[label=(\roman*)]
\item $ \nu $ besitzt eine Dichte bzgl $ \mu $
\item $ \forall A \in \calA$: $ \mu(A) = 0 \Longrightarrow \nu(A) = 0 $
\end{enumerate}

\end{karte}


\begin{karte}[Stochastik-I]{absolut-stetig \\ \& \\ singul�r}[2.4.25]

$ \XcalA $ mb Raum, Ma�e $\mu $ und $ \nu $:
\cmhalf
\begin{enumerate}[label=(\roman*)]
\item $ \nu $ besitzt eine Dichte bzgl $ \mu $
\item $ \forall A \in \calA $: $ \mu(A) = 0 \Longrightarrow \nu(A) = 0 $
\end{enumerate}

\end{karte}


\begin{karte}[Stochastik-I]{Zerlegungssatz von Lebesgue}[2.4.25]
$ \XcalA $ mb Raum, Ma�e $ \mu $ und $ \nu $:
\cmone
\begin{itemize}
\item $ \nu $ l�sst sich auf genau eine Weise als $ \nu = \nu_1 + \nu_2 $ darstellen
\item $ \nu_1 $ und $ \nu_2 $ sind Ma�e auf $ \XcalA $
\item $ \nu_1 \ll \mu $
\item $ \nu_2 \perp \mu $
\end{itemize}
	
\end{karte}


\begin{karte}[Stochastik-I]{kartesisches Produkt}[unter 2.5]

$ (X_i, \calA_i)_{1 \leq i \leq n} $ mb R�ume:
\cmone
$$
X := \prodd{i=1}{n}{X_i} \equiv \btimess{i=1}{n}{X_i}
$$

\end{karte}


\begin{karte}[Stochastik-I]{Projektionsabbildung}[unter 2.5]
\textit{
Alle Tupel werden auf ihre jeweils i-te Koordinate abgebildet, �hnlich einer orthogonalen Projektion auf eine Achse (aber formal anders). F�r dieses i werden alle $ x_i $ aus $ X_i $ angenommen.
}
\cmone
$$
p_i: X \ni (x_q, x_2, ..., x_n) \longrightarrow p_i((x_1, ... x_n)) := x_i \in X_i
$$

\end{karte}


\begin{karte}[Stochastik-I]{Produkt-$\sigma$-Algebra}[2.5.1]
Die kleinste $\sigma$-Algebra, bzgl der alle $ n $ Projektionen mb sind, auch genannt das Produkt der $\sigma$-Algebren $ \calA_1, ..., \calA_n $:
\cmhalf
$$
\botimess{i=1}{n}{\calA_i} := \calA_1 \times ... \times \calA_n := \sigma \left( \bcupp{i=1}{n}{p_i^{-1}} (\calA_i) \right) \equiv \sigma(p_i, ..., p_n)
$$	
\end{karte}


\begin{karte}[Stochastik-I]{DAS Produktma�}[2.5.3]

Das eind Produktma� $ \pi $ der Ma�e $ \mu_1, \mu_2, ... , \mu_n $ auf \\
$ \calA = \botimess{i=1}{n}{\calA_i} $ mit $ A_i \in \calA_I $ f�r
 $ 1 \leq i \leq n $ hat f�r $ n $ $ \sigma $-finite \\
 Ma�r�ume $ (X_i, \calA_i, \mu_i)_{1 \leq i \leq n} $ die folgende Eigenschaft und \\
 mit $ \pi \equiv \botimess{i=1}{n}{\mu_i} $ ergibt sich ein neuer Ma�raum:
$$
\pi (A_1 \times A_2 \times ... \times A_n) := \mu_1(A_1) \cdot \mu_2(A_2) \cdot ... \cdot \mu_n(A_n)
$$
$$
\left( \btimess{i=1}{n}{X_i}, \, \botimess{i=1}{n}{\calA_i}, \, \botimess{i=1}{n}{\mu_i} \right)
$$

\end{karte}


\begin{karte}[Stochastik-I]{Satz von Tonelli}[2.5.4]
	
$ (X_i, \calA_i, \mu_i) $ 2 $ \sigma $-finite Ma�r�ume, $ (X, \calA, \pi) $ Produktraum, eind best Produkt-Ma� $ \pi $, $ f \in \ZplusXcalAinf $ n-neg mb Funk:
\begin{align*}
(X_1, \calA_1) \ni x_1 \longrightarrow \inttt{}{}{f(x_1,x_2)}{\mu_2}{x_2} \in (\overline\R_\geq, \calB(\overline{\R}_\geq)\\
\intt{}{}{f}{\pi} = \int{}{}{\left( \inttt{}{}{f(x_1,x_2)}{\mu_2}{x_2} \right) }{\mu_1(}{x_1}) \\	
= \int{}{}{\left( \inttt{}{}{f(x_1,x_2)}{\mu_1}{x_1} \right) }{\mu_2(}{x_2})
\end{align*}

\end{karte}


\begin{karte}[Stochastik-I]{Satz von Fubini}[2.5.5]

$ (X_i, \calA_i, \mu_i) $ 2 $ \sigma $-finite Ma�r�ume, $ (X, \calA, \pi) $ der zug Produktraum mit dem eind best Produktma� $ \pi $, $ f \in \ZXcalAinf $ eine $ \pi $-integrierbare Funktion, dann gilt ebenfalls Tonelli.
\cmhalf
\textit
{Zus�tzlich wird also Integrierbarkeit bzgl des Produktma�es gefordert. Daf�r kann man bel, nicht nur n-neg Funktionen integrieren. Bei Tonelli reicht die Integrierbarkeit der iterierten Integrale aus. Bei der R�ckf�hrung mehr- auf eindimensionale Integrale ist die Reihenfolge bei beiden egal. 
}

\end{karte}

\end{document}